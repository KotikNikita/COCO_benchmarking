\documentclass{article}
%\documentclass[sigconf]{acmart}
\usepackage{url}
\usepackage{amsmath}
\usepackage{amssymb}
\usepackage{graphicx} % Required for inserting images

\title{Turbo}
\author{}
\date{December 2023}

\begin{document}

\maketitle

\section{Links}
\begin{itemize}
    \item Link to the Turbo Article: \url{https://papers.nips.cc/paper_files/paper/2019/file/6c990b7aca7bc7058f5e98ea909e924b-Paper.pdf}
    \item Explanation of GP regretion: https://thegradient.pub/gaussian-process-not-quite-for-dummies/
    \item https://digital.library.adelaide.edu.au/dspace/bitstream/2440/47972/8/02whole.pdf formulaa: https://digital.library.adelaide.edu.au/dspace/bitstream/2440/47972/8/02whole.pdf
    
\end{itemize}

\begin{abstract}
to be written
\end{abstract}

\section{Introduction}



\section{Algorithm Presentation}

The algorithm uses two strategies together to find global minima of functions. One is to focus on regions called trust regions. These evolve depending on the performance of the algorithm. The second is to approximate the objective function through a Gaussian process. In this way, the algorithm consists of an iteration of these two strategies that we explain in more detail below.


\subsection{Bayesian Optimization}

TuRBO uses a statistical model called a Gaussian process to approximate the objective function. This is a collection of random variables (possibly not enumerable) such that each finite subset follows a multivariate normal distribution. As in the finite case, this $\mathcal{GP}$ process is completely determined by a function $\mu$ and function $\Sigma$ that fulfill the role of the mean and the variance respectively.

To make predictions using this approximation, suppose we have some noisy observations of a function $f$ $$y(x) = f(x)+ \epsilon \sigma_y.$$ where $\epsilon \sim \mathcal{N}(0,1)$ ant the parameter $\sigma_y$ accounts for the amplitude of the noise. We can suppose that the probability of the observations are given by a Gaussian process where the mean is usually taken to be 0. This is, 
\begin{equation*}
    p(f(x)|\theta) = \mathcal{GP}(0,K(x,x') + I\sigma_y).
\end{equation*}
In this way we took $\Sigma$ to be some function $K$ called the kernel and the identity times $\sigma_y$ that accounts for the noise in the observations. Several choices of kernel are possible. In the case of turbo the a Matérn-5/2 kernel with ARD covariance function is used. This function gives the covariance of two points as a function of its distance and its given by

$$
K(x, x' ; \zeta, \nu, \lambda_1 \dots \lambda_d)=\frac{s^2}{2^{\nu-1} \Gamma(\nu)}\left(\frac{d(x, x' ; \lambda_1 \dots \lambda_d)}{\zeta}\right)^\nu K_\nu\left(\frac{d(x, x' ; \lambda_1 \dots \lambda_d)}{\zeta}\right)
$$
In the formula the parameter $\nu $ is a parameter that accounts for the differentiability of the learned function. In our case $\nu$ is taken to be $5/2$ meaning that the learned function will be two times differentiable. $d$ is the distance between $x$ and $x'$ by doing a res-scaling according to $\lambda_{i}$. Additionally the parameters $s^2$ and $\zeta$ are the variance and another scaling factor. Finally $K_{\nu}$ is the modified Bessel function. 

 We can maximize the log likelihood function to estimate the parameters of the $\mathcal{GP}$. In the article the authors seated bounds for the hyper-parameters as follows : $\lambda_i \in [0.005,2.0]$, $s^2 \in [0.05, 20.0]$ and $\sigma^2_y \in [0.0005,0.1]$


 

\subsection{Trust region methods}

A trust region method is an optimization method that is limited to a subset of Omega (TR) in which the objective function is approximated with another function with better characteristics. The TR is updated iteratively in such a way that if our prediction is good, the TR grows and if it is bad it shrinks. In this way, we ensure that if the approximation of the objective function is good, the search area can be expanded and if it is bad, it should be decreased in order to obtain a better approximation of the objective function.

In TuRBO methodology, a subset of $q$ points from our function in the trust regions is sampled. These points are then evaluated against the current best approximation. Their performance is categorized as either successes or failures based on whether they improve upon the previous minimum. This categorization drives the adjustment of parameters $\tau_{\text{fail}}$ and $\tau_{\text{succ}}$: if failures outnumber $\tau_{\text{fail}}$, the sampling area decreases; conversely, if successes surpass $\tau_{\text{succ}}$, the sampling area expands. 

\subsection{TuRBO algorithm}

The algorithm consists of using m trust regions with m independent GP models. In each of the trust regions we set $L \longleftarrow L_{\text{init}}$. The trust region will be then an hyper rectangle of volume $L^d$. Each side of the hyper-rectangle is calculated using the parameters $\lambda_i$ of the respective $\mathcal{GP}$ of the trust region. in this way we have $L_i=\lambda_i L /\left(\prod_{j=1}^d \lambda_j\right)^{1 / d}$. After this we do the sampling of the $q$ point within the trust region and we evaluate the performance of the algorithm. The update of the trust region is done by setting $L \leftarrow \min \left\{L_{\max }, 2 L\right\}$ if we decide to increase the trust region, or $L \leftarrow  L/2$ if we shrink it. In the case $L < L_{\min}$ we discard the trust region and create another one. 

Finally to solve the problem of how to sample our badge of $q$ pints from the trust regions in each iteration TuRBO uses Thomson sampling to allocate the points in an efficient way as follows:




\subsection{parameters of the algorithm}
\begin{itemize}
    \item $L_{\text{max}}$: Maximum size of the side of the hyperrectangle
    \item $L_{\text{min}}$: Minimum size of the side of the hyperrectangle
    \item $L_{\text{init}}$: This magnitude is used to calculate the sides of the TR hyperrectangle. The true lengths are calculated such that the initial volume is $L_{\text{init}}^d$ and are scaled using the $\mathcal{GP}$ $\lambda_i$ parameters
    \item $\tau_{\text{succ}}$: If this number of successes is exceeded, the trust region is expanded.
    \item $\tau_{\text{fail}}$: If this number of failures is exceeded, the trust region is shrunk.
\end{itemize}

\section{questions}

\begin{enumerate}
    \item It seems that the algorithm does the optimization process using several regions at the same time according to the introduction, but in the description of the algorithm only one is mentioned. The algorithm consists of doing the same thing in several regions at the same time?
    \item Do we include the theory of the Gaussian process? Or are we going to see it in class?
    \item  I don't understand the change of X when the observation has noise
    \item is the formula for the kernell
    correct?

\end{enumerate}

\section{Experimental Procedure}

\maketitle


\end{document}


% include preamble:
\include{templatepreamble.tex}

%%%%%%%%%%%%%%%%%%%%%%%%%%%%%%%%%%%%%%%%%%%%%%%%%%%%%%%%%%%%%%%%%%%%%%%%%%%%%%%
%%%%%%%%% TO BE EDITED %%%%%%%%%%%%%%%%%%%%%%%%%%%%%%%%%%%%%%%%%%%%%%%%%%%%%%%%
%%%%%%%%%%%%%%%%%%%%%%%%%%%%%%%%%%%%%%%%%%%%%%%%%%%%%%%%%%%%%%%%%%%%%%%%%%%%%%%

% Algorithm names as they appear in the tables, uncomment and adapt if necessary
% \newcommand{\algAtables}{ALGO1}  % first argument in the post-processing
% \newcommand{\algBtables}{ALGO2}  % second argument in the post-processing
% \newcommand{\algCtables}{ALGO3}  % third argument in the post-processing
% \newcommand{\algDtables}{ALGO4}  % forth argument in the post-processing
% ...
% location of pictures files
\newcommand{\bbobdatapath}{ppdata/} % change default output folder of COCO if desired



%%%%%%%%%%%%%%%%%%%%%%%%%%%%%%%%%%%%%%%%%%%%%%%%%%%%%%%%%%%%%%%%%%%%%%%%%%%%%%%
% read in data and deal with the different number of algorithms:
\input{\bbobdatapath cocopp_commands.tex}

\ifthenelse{\equal{\numofalgs}{1}}{
   \graphicspath{{\bbobdatapath\algfolder}}}{
	 \graphicspath{{\bbobdatapath\algsfolder}}
}

\ifthenelse{\isundefined{\algorithmA}}{\newcommand{\algorithmA}{\algname}}{}
%\ifthenelse{\isundefined{\algorithmA}{\newcommand{\algorithmA}{\change{MY-ALGORITHM-NAME}}}{}  % better use the previous line?
%%



%%%%%%%%%%%%%%%%%%%%%%%%%%%%%%%%%%%%%%%%%%%%%%%%%%%%%%%%%%%%%%%%%%%%%%%%%%%%%%%
%%%%%%%%%%%%%%%%%%%%%%%%%%%%%%%%%%%%%%%%%%%%%%%%%%%%%%%%%%%%%%%%%%%%%%%%%%%%%%%
%%%%%%%%%%%%%%%%%%%%%%%%%%%%%%%%%%%%%%%%%%%%%%%%%%%%%%%%%%%%%%%%%%%%%%%%%%%%%%%

\begin{document}


\title{Black-Box Optimization Benchmarking Template for the Comparison of Algorithms on the \bbobls Testbed}
\renewcommand{\shorttitle}{Template to Compare Algorithms on the \bbobls Testbed}
\subtitle{Draft version}



\author{Firstname Lastname}
%\authornote{tba if needed}
%\orcid{1234-5678-9012}
%\affiliation{%
%  \institution{Institute for Clarity in Documentation}
%  \streetaddress{P.O. Box 1212}
%  \city{Dublin} 
%  \state{Ohio} 
%  \postcode{43017-6221}
%}
%\email{trovato@corporation.com}
%
%\author{G.K.M. Tobin}
%\authornote{The secretary disavows any knowledge of this author's actions.}
%\affiliation{%
%  \institution{Institute for Clarity in Documentation}
%  \streetaddress{P.O. Box 1212}
%  \city{Dublin} 
%  \state{Ohio} 
%  \postcode{43017-6221}
%}
%\email{webmaster@marysville-ohio.com}
%
%\author{Lars Th{\o}rv{\"a}ld}
%\authornote{This author is the
%  one who did all the really hard work.}
%\affiliation{%
%  \institution{The Th{\o}rv{\"a}ld Group}
%  \streetaddress{1 Th{\o}rv{\"a}ld Circle}
%  \city{Hekla} 
%  \country{Iceland}}
%\email{larst@affiliation.org}
%
%\author{Lawrence P. Leipuner}
%\affiliation{
%  \institution{Brookhaven Laboratories}
%  \streetaddress{P.O. Box 5000}}
%\email{lleipuner@researchlabs.org}
%
%\author{Sean Fogarty}
%\affiliation{%
%  \institution{NASA Ames Research Center}
%  \city{Moffett Field}
%  \state{California} 
%  \postcode{94035}}
%\email{fogartys@amesres.org}
%
%\author{Charles Palmer}
%\affiliation{%
%  \institution{Palmer Research Laboratories}
%  \streetaddress{8600 Datapoint Drive}
%  \city{San Antonio}
%  \state{Texas} 
%  \postcode{78229}}
%\email{cpalmer@prl.com}
%
%\author{John Smith}
%\affiliation{\institution{The Th{\o}rv{\"a}ld Group}}
%\email{jsmith@affiliation.org}
%
%\author{Julius P.~Kumquat}
%\affiliation{\institution{The Kumquat Consortium}}
%\email{jpkumquat@consortium.net}

% The default list of authors is too long for headers}
\renewcommand{\shortauthors}{Firstname Lastname et. al.}



\begin{abstract}
to be written
\end{abstract}


%
% The code below should be generated by the tool at
% http://dl.acm.org/ccs.cfm
% Please copy and paste the code instead of the example below. 
%
 \begin{CCSXML}
<ccs2012>
<concept>
<concept_id>10010147.10010178.10010205.10010208</concept_id>
<concept_desc>Computing methodologies~Continuous space search</concept_desc>
<concept_significance>500</concept_significance>
</concept>
</ccs2012>
\end{CCSXML}

\ccsdesc[500]{Computing methodologies~Continuous space search}


% We no longer use \terms command
%\terms{Algorithms}

% Complete with anything that is needed
\keywords{Benchmarking, Black-box optimization, Large scale optimization}

\maketitle


% \section{Introduction}
%
% \section{Algorithm Presentation}
%
% \section{Experimental Procedure}
%

%%%%%%%%%%%%%%%%%%%%%%%%%%%%%%%%%%%%%%%%%%%%%%%%%%%%%%%%%%%%%%%%%%%%%%%%%%%%%%%
\section{CPU Timing}
%%%%%%%%%%%%%%%%%%%%%%%%%%%%%%%%%%%%%%%%%%%%%%%%%%%%%%%%%%%%%%%%%%%%%%%%%%%%%%%
% note that the following text is just a proposal and can/should be changed to your needs:
In order to evaluate the CPU timing of the algorithm, we have run the \change{\algorithmA} with restarts on the entire \bbobls test suite \cite{varelas2020benchmarking} for $2 D$ function evaluations according to \cite{hansen2016exp}. The \change{C/Java/Matlab/Octave/Python} code was run on a \change{Mac Intel(R) Core(TM) i5-2400S CPU @ 2.50GHz} with \change{1} processor and \change{4} cores \change{and (compile) options xxx}. The time per function evaluation for dimensions 20, 40, 80, 160, 320\change{, 640} equals \change{$x.x$}, \change{$x.x$}, \change{$x.x$}, \change{$xx$}, \change{$xxx$}\change{, and $xxx$} seconds respectively. 

\ifthenelse{\equal{\numofalgs}{1}}{}{
\change{repeat the above for any algorithm tested}
}

%%%%%%%%%%%%%%%%%%%%%%%%%%%%%%%%%%%%%%%%%%%%%%%%%%%%%%%%%%%%%%%%%%%%%%%%%%%%%%%
\section{Results}
%%%%%%%%%%%%%%%%%%%%%%%%%%%%%%%%%%%%%%%%%%%%%%%%%%%%%%%%%%%%%%%%%%%%%%%%%%%%%%%

Results from experiments according to \cite{hansen2016exp} and \cite{hansen2022perfass} on the
benchmark functions given in \cite{varelas2020benchmarking} are
presented in
%%
\ifthenelse{\equal{\numofalgs}{1}}{
Figures~\ref{fig:ERTgraphs}, \ref{fig:ECDFs}, and \ref{fig:ECDFsingleOne} and Tables~\ref{tab:ERTs80} and \ref{tab:ERTs320}.
}{\ifthenelse{\equal{\numofalgs}{2}}{
Figures~\ref{fig:scaling}, \ref{fig:scatterplots}, \ref{fig:ECDFs80D}, \ref{fig:ECDFs320D} and \ref{fig:ECDFsingleOne}, and Tables~\ref{tab:ERTs80} and \ref{tab:ERTs320}.
}{\ifthenelse{\(\numofalgs > 2\)}{
Figures~\ref{fig:scaling}, \ref{fig:ECDFs80D}, \ref{fig:ECDFs320D}, and \ref{fig:ECDFsingleOne} and Tables~\ref{tab:ERTs80} and \ref{tab:ERTs320}.
}{}}}
%%
The experiments were performed with COCO \cite{hansen2020cocoplat}, version
\change{\version}, the plots were produced with version \change{\version}.

The \textbf{expected runtime (ERT)}, used in the figures and tables,
depends on a given target function value, $\ftarget=\fopt+\Df$, and is
computed over all relevant trials as the number of function
evaluations executed during each trial while the best function value
did not reach \ftarget, summed over all trials and divided by the
number of trials that actually reached \ftarget\
\cite{hansen2012exp,price1997dev}. 
\textbf{Statistical significance} is tested with the rank-sum test for a given
target $\Delta\ftarget$ using, for each trial,
either the number of needed function evaluations to reach
$\Delta\ftarget$ (inverted and multiplied by $-1$), or, if the target
was not reached, the best $\Df$-value achieved, measured only up to
the smallest number of overall function evaluations for any
unsuccessful trial under consideration.


\ifthenelse{\equal{\numofalgs}{1}}{


%%%%%%%%%%%%%%%%%%%%%%%%%%%%%%%%%%%%%%%%%%%%%%%%%%%%%%%%%%%%%%%%%%%%%%%%%%%%%%%

% Scaling of ERT with dimension

%%%%%%%%%%%%%%%%%%%%%%%%%%%%%%%%%%%%%%%%%%%%%%%%%%%%%%%%%%%%%%%%%%%%%%%%%%%%%%%
\begin{figure*}
\begin{tabular}{l@{\hspace*{-0.0\textwidth}}l@{\hspace*{-0.0\textwidth}}l@{\hspace*{-0.0\textwidth}}l}
\includegraphics[width=0.24\textwidth]{ppfigdim_f001}&
\includegraphics[width=0.24\textwidth]{ppfigdim_f002}&
\includegraphics[width=0.24\textwidth]{ppfigdim_f003}&
\includegraphics[width=0.24\textwidth]{ppfigdim_f004}\\[-1ex]
\includegraphics[width=0.24\textwidth]{ppfigdim_f005}&
\includegraphics[width=0.24\textwidth]{ppfigdim_f006}&
\includegraphics[width=0.24\textwidth]{ppfigdim_f007}&
\includegraphics[width=0.24\textwidth]{ppfigdim_f008}\\[-1ex]
\includegraphics[width=0.24\textwidth]{ppfigdim_f009}&
\includegraphics[width=0.24\textwidth]{ppfigdim_f010}&
\includegraphics[width=0.24\textwidth]{ppfigdim_f011}&
\includegraphics[width=0.24\textwidth]{ppfigdim_f012}\\[-1ex]
\includegraphics[width=0.24\textwidth]{ppfigdim_f013}&
\includegraphics[width=0.24\textwidth]{ppfigdim_f014}&
\includegraphics[width=0.24\textwidth]{ppfigdim_f015}&
\includegraphics[width=0.24\textwidth]{ppfigdim_f016}\\[-1ex]
\includegraphics[width=0.24\textwidth]{ppfigdim_f017}&
\includegraphics[width=0.24\textwidth]{ppfigdim_f018}&
\includegraphics[width=0.24\textwidth]{ppfigdim_f019}&
\includegraphics[width=0.24\textwidth]{ppfigdim_f020}\\[-1ex]
\includegraphics[width=0.24\textwidth]{ppfigdim_f021}&
\includegraphics[width=0.24\textwidth]{ppfigdim_f022}&
\includegraphics[width=0.24\textwidth]{ppfigdim_f023}&
\includegraphics[width=0.24\textwidth]{ppfigdim_f024}
\end{tabular}
\vspace{-3ex}
 \caption{\label{fig:ERTgraphs}
 \bbobppfigdimlegend{$f_1$ and $f_{24}$}
 }
\end{figure*}

%%%%%%%%%%%%%%%%%%%%%%%%%%%%%%%%%%%%%%%%%%%%%%%%%%%%%%%%%%%%%%%%%%%%%%%%%%%%%%%
%%%%%%%%%%%%%%%%%%%%%%%%%%%%%%%%%%%%%%%%%%%%%%%%%%%%%%%%%%%%%%%%%%%%%%%%%%%%%%%
 
% Table showing the expected runtime (ERT in number of function
% evaluations) for functions $f_1$--$f_{24} for dimension 80$.

%%%%%%%%%%%%%%%%%%%%%%%%%%%%%%%%%%%%%%%%%%%%%%%%%%%%%%%%%%%%%%%%%%%%%%%%%%%%%%%


\begin{table*}\tiny
%\hfill80-D\hfill~\\[1ex]
{\normalsize \color{red}
\ifthenelse{\isundefined{\algorithmG}}{}{more than 6 algorithms: please split the tables below by hand until it fits to the page limits}
}
\mbox{\begin{minipage}[t]{0.499\textwidth}\tiny
\centering
\pptableheader 

\input{\bbobdatapath\algfolder pptable_f001_80D} 
\input{\bbobdatapath\algfolder pptable_f002_80D}
\input{\bbobdatapath\algfolder pptable_f003_80D}
\input{\bbobdatapath\algfolder pptable_f004_80D}
\input{\bbobdatapath\algfolder pptable_f005_80D}
\input{\bbobdatapath\algfolder pptable_f006_80D}
\input{\bbobdatapath\algfolder pptable_f007_80D}
\input{\bbobdatapath\algfolder pptable_f008_80D}
\input{\bbobdatapath\algfolder pptable_f009_80D}
\input{\bbobdatapath\algfolder pptable_f010_80D}
\input{\bbobdatapath\algfolder pptable_f011_80D}
\input{\bbobdatapath\algfolder pptable_f012_80D}

\pptablefooter

\end{minipage}
\hspace{0.002\textwidth}
\begin{minipage}[t]{0.499\textwidth}\tiny
\centering

\pptableheader 

\input{\bbobdatapath\algfolder pptable_f013_80D}
\input{\bbobdatapath\algfolder pptable_f014_80D}
\input{\bbobdatapath\algfolder pptable_f015_80D}
\input{\bbobdatapath\algfolder pptable_f016_80D}
\input{\bbobdatapath\algfolder pptable_f017_80D}
\input{\bbobdatapath\algfolder pptable_f018_80D}
\input{\bbobdatapath\algfolder pptable_f019_80D}
\input{\bbobdatapath\algfolder pptable_f020_80D}
\input{\bbobdatapath\algfolder pptable_f021_80D}
\input{\bbobdatapath\algfolder pptable_f022_80D}
\input{\bbobdatapath\algfolder pptable_f023_80D}
\input{\bbobdatapath\algfolder pptable_f024_80D}

\pptablefooter
\end{minipage}}

\caption[Table of ERTs]{\label{tab:ERTs80}\bbobpptablecaption{dimension $80$} \cocoversion
}
\end{table*}

%%%%%%%%%%%%%%%%%%%%%%%%%%%%%%%%%%%%%%%%%%%%%%%%%%%%%%%%%%%%%%%%%%%%%%%%%%%%%%%
%%%%%%%%%%%%%%%%%%%%%%%%%%%%%%%%%%%%%%%%%%%%%%%%%%%%%%%%%%%%%%%%%%%%%%%%%%%%%%%
 
% Table showing the expected runtime (ERT in number of function
% evaluations) for functions $f_1$--$f_{24} for dimension 320$.

%%%%%%%%%%%%%%%%%%%%%%%%%%%%%%%%%%%%%%%%%%%%%%%%%%%%%%%%%%%%%%%%%%%%%%%%%%%%%%%
\begin{table*}\tiny
%\hfill320-D\hfill~\\[1ex]
\mbox{\begin{minipage}[t]{0.499\textwidth}\tiny
\centering
\pptableheader 

\input{\bbobdatapath\algfolder pptable_f001_320D} 
\input{\bbobdatapath\algfolder pptable_f002_320D}
\input{\bbobdatapath\algfolder pptable_f003_320D}
\input{\bbobdatapath\algfolder pptable_f004_320D}
\input{\bbobdatapath\algfolder pptable_f005_320D}
\input{\bbobdatapath\algfolder pptable_f006_320D}
\input{\bbobdatapath\algfolder pptable_f007_320D}
\input{\bbobdatapath\algfolder pptable_f008_320D}
\input{\bbobdatapath\algfolder pptable_f009_320D}
\input{\bbobdatapath\algfolder pptable_f010_320D}
\input{\bbobdatapath\algfolder pptable_f011_320D}
\input{\bbobdatapath\algfolder pptable_f012_320D}

\pptablefooter

\end{minipage}
\hspace{0.002\textwidth}
\begin{minipage}[t]{0.499\textwidth}\tiny
\centering
\pptableheader 

\input{\bbobdatapath\algfolder pptable_f013_320D}
\input{\bbobdatapath\algfolder pptable_f014_320D}
\input{\bbobdatapath\algfolder pptable_f015_320D}
\input{\bbobdatapath\algfolder pptable_f016_320D}
\input{\bbobdatapath\algfolder pptable_f017_320D}
\input{\bbobdatapath\algfolder pptable_f018_320D}
\input{\bbobdatapath\algfolder pptable_f019_320D}
\input{\bbobdatapath\algfolder pptable_f020_320D}
\input{\bbobdatapath\algfolder pptable_f021_320D}
\input{\bbobdatapath\algfolder pptable_f022_320D}
\input{\bbobdatapath\algfolder pptable_f023_320D}
\input{\bbobdatapath\algfolder pptable_f024_320D}

\pptablefooter

\end{minipage}}
 
\caption[Table of ERTs]{\label{tab:ERTs320}\bbobpptablecaption{dimension $320$} \cocoversion
}
\end{table*}

%%%%%%%%%%%%%%%%%%%%%%%%%%%%%%%%%%%%%%%%%%%%%%%%%%%%%%%%%%%%%%%%%%%%%%%%%%%%%%%


%%%%%%%%%%%%%%%%%%%%%%%%%%%%%%%%%%%%%%%%%%%%%%%%%%%%%%%%%%%%%%%%%%%%%%%%%%%%%%%
%%%%%%%%%%%%%%%%%%%%%%%%%%%%%%%%%%%%%%%%%%%%%%%%%%%%%%%%%%%%%%%%%%%%%%%%%%%%%%%

% Empirical cumulative distribution functions (ECDFs) per function group.

%%%%%%%%%%%%%%%%%%%%%%%%%%%%%%%%%%%%%%%%%%%%%%%%%%%%%%%%%%%%%%%%%%%%%%%%%%%%%%%

\begin{figure*}
\begin{tabular}{l@{\hspace*{-0.00\textwidth}}l@{\hspace*{0.01\textwidth}}|l@{\hspace*{-0.00\textwidth}}l}
\multicolumn{2}{c}{$D=80$} & \multicolumn{2}{c}{$D=320$}\\[-0.5ex]
\rot[3]{all functions}
\includegraphics[width=0.2362\textwidth]{pprldistr_80D_noiselessall} &
\includegraphics[width=0.2362\textwidth]{ppfvdistr_80D_noiselessall} &
\includegraphics[width=0.2362\textwidth]{pprldistr_320D_noiselessall} &
\includegraphics[width=0.2362\textwidth]{ppfvdistr_320D_noiselessall} \\[-0.2em]
\rot[2.9]{separable fcts}
\includegraphics[width=0.2362\textwidth]{pprldistr_80D_separ} &
\includegraphics[width=0.2362\textwidth]{ppfvdistr_80D_separ} &
\includegraphics[width=0.2362\textwidth]{pprldistr_320D_separ} &
\includegraphics[width=0.2362\textwidth]{ppfvdistr_320D_separ} \\[-0.2em]
\rot[1.45]{misc.\ moderate fcts}
\includegraphics[width=0.2362\textwidth]{pprldistr_80D_lcond} &
\includegraphics[width=0.2362\textwidth]{ppfvdistr_80D_lcond} &
\includegraphics[width=0.2362\textwidth]{pprldistr_320D_lcond} &
\includegraphics[width=0.2362\textwidth]{ppfvdistr_320D_lcond} \\[-0.2em]
\rot[1.5]{ill-conditioned fcts}
\includegraphics[width=0.2362\textwidth]{pprldistr_80D_hcond} &
\includegraphics[width=0.2362\textwidth]{ppfvdistr_80D_hcond} &
\includegraphics[width=0.2362\textwidth]{pprldistr_320D_hcond} &
\includegraphics[width=0.2362\textwidth]{ppfvdistr_320D_hcond} \\[-0.2em]
\rot[2.3]{multi-modal fcts}
\includegraphics[width=0.2362\textwidth]{pprldistr_80D_multi} &
\includegraphics[width=0.2362\textwidth]{ppfvdistr_80D_multi} &
\includegraphics[width=0.2362\textwidth]{pprldistr_320D_multi} &
\includegraphics[width=0.2362\textwidth]{ppfvdistr_320D_multi} \\[-0.2em]
\rot[1.7]{weak structure fcts}
\includegraphics[width=0.2362\textwidth]{pprldistr_80D_mult2} &
\includegraphics[width=0.2362\textwidth]{ppfvdistr_80D_mult2} &
\includegraphics[width=0.2362\textwidth]{pprldistr_320D_mult2} &
\includegraphics[width=0.2362\textwidth]{ppfvdistr_320D_mult2}
\vspace*{-1ex}
\end{tabular}
 \caption{\label{fig:ECDFs}
 \bbobpprldistrlegend{}
 }
\end{figure*}

%%%%%%%%%%%%%%%%%%%%%%%%%%%%%%%%%%%%%%%%%%%%%%%%%%%%%%%%%%%%%%%%%%%%%%%%%%%%%%%
%%%%%%%%%%%%%%%%%%%%%%%%%%%%%%%%%%%%%%%%%%%%%%%%%%%%%%%%%%%%%%%%%%%%%%%%%%%%%%%

% ECDFs per function

%%%%%%%%%%%%%%%%%%%%%%%%%%%%%%%%%%%%%%%%%%%%%%%%%%%%%%%%%%%%%%%%%%%%%%%%%%%%%%%
\begin{figure*}
\centering
\begin{tabular}{l@{\hspace*{-0.00\textwidth}}l@{\hspace*{0.01\textwidth}}|l@{\hspace*{-0.00\textwidth}}l}
\includegraphics[width=0.24\textwidth]{pprldmany-single-functions/pprldmany_f001}&
\includegraphics[width=0.24\textwidth]{pprldmany-single-functions/pprldmany_f002}&
\includegraphics[width=0.24\textwidth]{pprldmany-single-functions/pprldmany_f003}&
\includegraphics[width=0.24\textwidth]{pprldmany-single-functions/pprldmany_f004}\\[-0.2em]
\includegraphics[width=0.24\textwidth]{pprldmany-single-functions/pprldmany_f005}&
\includegraphics[width=0.24\textwidth]{pprldmany-single-functions/pprldmany_f006}&
\includegraphics[width=0.24\textwidth]{pprldmany-single-functions/pprldmany_f007}&
\includegraphics[width=0.24\textwidth]{pprldmany-single-functions/pprldmany_f008}\\[-0.2em]
\includegraphics[width=0.24\textwidth]{pprldmany-single-functions/pprldmany_f009}&
\includegraphics[width=0.24\textwidth]{pprldmany-single-functions/pprldmany_f010}&
\includegraphics[width=0.24\textwidth]{pprldmany-single-functions/pprldmany_f011}&
\includegraphics[width=0.24\textwidth]{pprldmany-single-functions/pprldmany_f012}\\[-0.2em]
\includegraphics[width=0.24\textwidth]{pprldmany-single-functions/pprldmany_f013}&
\includegraphics[width=0.24\textwidth]{pprldmany-single-functions/pprldmany_f014}&
\includegraphics[width=0.24\textwidth]{pprldmany-single-functions/pprldmany_f015}&
\includegraphics[width=0.24\textwidth]{pprldmany-single-functions/pprldmany_f016}\\[-0.2em]
\includegraphics[width=0.24\textwidth]{pprldmany-single-functions/pprldmany_f017}&
\includegraphics[width=0.24\textwidth]{pprldmany-single-functions/pprldmany_f018}&
\includegraphics[width=0.24\textwidth]{pprldmany-single-functions/pprldmany_f019}&
\includegraphics[width=0.24\textwidth]{pprldmany-single-functions/pprldmany_f020}\\[-0.2em]
\includegraphics[width=0.24\textwidth]{pprldmany-single-functions/pprldmany_f021}&
\includegraphics[width=0.24\textwidth]{pprldmany-single-functions/pprldmany_f022}&
\includegraphics[width=0.24\textwidth]{pprldmany-single-functions/pprldmany_f023}&
\includegraphics[width=0.24\textwidth]{pprldmany-single-functions/pprldmany_f024}
\vspace*{-1ex}
\end{tabular}
 \caption{\label{fig:ECDFsingleOne}
	\bbobecdfcaptionsinglefunctionssingledim{$\!\!$s 20 to 640}
}
\end{figure*}



%%%%%%%%%%%%%%%%%%%%%%%%%%%%%%%%%%%%%%%%%%%%%%%%%%%%%%%%%%%%%%%%%%%%%%%%%%%%%%%

}{} % end of 1 algorithm template





\ifthenelse{\numofalgs > 1}{

%%%%%%%%%%%%%%%%%%%%%%%%%%%%%%%%%%%%%%%%%%%%%%%%%%%%%%%%%%%%%%%%%%%%%%%%%%%%%%%
%%%%%%%%%%%%%%%%%%%%%%%%%%%%%%%%%%%%%%%%%%%%%%%%%%%%%%%%%%%%%%%%%%%%%%%%%%%%%%%

% Scaling of ERT with dimension

%%%%%%%%%%%%%%%%%%%%%%%%%%%%%%%%%%%%%%%%%%%%%%%%%%%%%%%%%%%%%%%%%%%%%%%%%%%%%%%

\begin{figure*}
\centering
\begin{tabular}{@{}c@{}c@{}c@{}c@{}}
\includegraphics[width=0.24\textwidth]{ppfigs_f001}&
\includegraphics[width=0.24\textwidth]{ppfigs_f002}&
\includegraphics[width=0.24\textwidth]{ppfigs_f003}&
\includegraphics[width=0.24\textwidth]{ppfigs_f004}\\[-0.25em]
\includegraphics[width=0.24\textwidth]{ppfigs_f005}&
\includegraphics[width=0.24\textwidth]{ppfigs_f006}&
\includegraphics[width=0.24\textwidth]{ppfigs_f007}&
\includegraphics[width=0.24\textwidth]{ppfigs_f008}\\[-0.25em]
\includegraphics[width=0.24\textwidth]{ppfigs_f009}&
\includegraphics[width=0.24\textwidth]{ppfigs_f010}&
\includegraphics[width=0.24\textwidth]{ppfigs_f011}&
\includegraphics[width=0.24\textwidth]{ppfigs_f012}\\[-0.25em]
\includegraphics[width=0.24\textwidth]{ppfigs_f013}&
\includegraphics[width=0.24\textwidth]{ppfigs_f014}&
\includegraphics[width=0.24\textwidth]{ppfigs_f015}&
\includegraphics[width=0.24\textwidth]{ppfigs_f016}\\[-0.25em]
\includegraphics[width=0.24\textwidth]{ppfigs_f017}&
\includegraphics[width=0.24\textwidth]{ppfigs_f018}&
\includegraphics[width=0.24\textwidth]{ppfigs_f019}&
\includegraphics[width=0.24\textwidth]{ppfigs_f020}\\[-0.25em]
\includegraphics[width=0.24\textwidth]{ppfigs_f021}&
\includegraphics[width=0.24\textwidth]{ppfigs_f022}&
\includegraphics[width=0.24\textwidth]{ppfigs_f023}&
\includegraphics[width=0.24\textwidth]{ppfigs_f024}
\end{tabular}
\vspace*{-0.2cm}
\caption[Expected running time (\ERT) divided by dimension
versus dimension in log-log presentation]{
\label{fig:scaling}
\bbobppfigslegend{$f_1$ and $f_{24}$}. 
}
% 
\end{figure*}






%%%%%%%%%%%%%%%%%%%%%%%%%%%%%%%%%%%%%%%%%%%%%%%%%%%%%%%%%%%%%%%%%%%%%%%%%%%%%%%
%%%%%%%%%%%%%%%%%%%%%%%%%%%%%%%%%%%%%%%%%%%%%%%%%%%%%%%%%%%%%%%%%%%%%%%%%%%%%%%
 
% Empirical cumulative distribution functions (ECDFs) per function group
% for dimensions 80 and 320

%%%%%%%%%%%%%%%%%%%%%%%%%%%%%%%%%%%%%%%%%%%%%%%%%%%%%%%%%%%%%%%%%%%%%%%%%%%%%%%

\begin{figure*}
 \begin{tabular}{@{}c@{\hspace*{0.05\textwidth}}c@{}}
 separable fcts & moderate fcts \\
 \includeperfprof{pprldmany_80D_separ} &
 \includeperfprof{pprldmany_80D_lcond} \\ 
ill-conditioned fcts & multi-modal fcts \\
 \includeperfprof{pprldmany_80D_hcond} &
 \includeperfprof{pprldmany_80D_multi} \\ 
 weakly structured multi-modal fcts & all functions\\
 \includeperfprof{pprldmany_80D_mult2} & 
 \includeperfprof{pprldmany_80D_noiselessall} 
 \end{tabular}
\caption{
\label{fig:ECDFs80D}
\bbobECDFslegend{80}
}
\end{figure*}


\begin{figure*}
 \begin{tabular}{@{}c@{\hspace*{0.05\textwidth}}c@{}}
 separable fcts & moderate fcts \\
 \includeperfprof{pprldmany_320D_separ} &
 \includeperfprof{pprldmany_320D_lcond} \\ 
ill-conditioned fcts & multi-modal fcts \\
 \includeperfprof{pprldmany_320D_hcond} &
 \includeperfprof{pprldmany_320D_multi} \\ 
 weakly structured multi-modal fcts & all functions\\
 \includeperfprof{pprldmany_320D_mult2} & 
 \includeperfprof{pprldmany_320D_noiselessall} 
 \end{tabular}
\caption{
\label{fig:ECDFs320D}
\bbobECDFslegend{320}
}
\end{figure*}

%%%%%%%%%%%%%%%%%%%%%%%%%%%%%%%%%%%%%%%%%%%%%%%%%%%%%%%%%%%%%%%%%%%%%%%%%%%%%%%
%%%%%%%%%%%%%%%%%%%%%%%%%%%%%%%%%%%%%%%%%%%%%%%%%%%%%%%%%%%%%%%%%%%%%%%%%%%%%%%

% ECDFs per function in dimension 160

%%%%%%%%%%%%%%%%%%%%%%%%%%%%%%%%%%%%%%%%%%%%%%%%%%%%%%%%%%%%%%%%%%%%%%%%%%%%%%%
\begin{figure*}
\centering
\begin{tabular}{@{}l@{}l@{}l@{}l@{}l@{}}
\includegraphics[width=0.2\textwidth]{pprldmany-single-functions/pprldmany_f001_160D}&
\includegraphics[width=0.2\textwidth]{pprldmany-single-functions/pprldmany_f002_160D}&
\includegraphics[width=0.2\textwidth]{pprldmany-single-functions/pprldmany_f003_160D}&
\includegraphics[width=0.2\textwidth]{pprldmany-single-functions/pprldmany_f004_160D}\\
\includegraphics[width=0.2\textwidth]{pprldmany-single-functions/pprldmany_f005_160D}&
\includegraphics[width=0.2\textwidth]{pprldmany-single-functions/pprldmany_f006_160D}&
\includegraphics[width=0.2\textwidth]{pprldmany-single-functions/pprldmany_f007_160D}&
\includegraphics[width=0.2\textwidth]{pprldmany-single-functions/pprldmany_f008_160D}\\
\includegraphics[width=0.2\textwidth]{pprldmany-single-functions/pprldmany_f009_160D}&
\includegraphics[width=0.2\textwidth]{pprldmany-single-functions/pprldmany_f010_160D}&
\includegraphics[width=0.2\textwidth]{pprldmany-single-functions/pprldmany_f011_160D}&
\includegraphics[width=0.2\textwidth]{pprldmany-single-functions/pprldmany_f012_160D}\\
\includegraphics[width=0.2\textwidth]{pprldmany-single-functions/pprldmany_f013_160D}&
\includegraphics[width=0.2\textwidth]{pprldmany-single-functions/pprldmany_f014_160D}&
\includegraphics[width=0.2\textwidth]{pprldmany-single-functions/pprldmany_f015_160D}&
\includegraphics[width=0.2\textwidth]{pprldmany-single-functions/pprldmany_f016_160D}\\
\includegraphics[width=0.2\textwidth]{pprldmany-single-functions/pprldmany_f017_160D}&
\includegraphics[width=0.2\textwidth]{pprldmany-single-functions/pprldmany_f018_160D}&
\includegraphics[width=0.2\textwidth]{pprldmany-single-functions/pprldmany_f019_160D}&
\includegraphics[width=0.2\textwidth]{pprldmany-single-functions/pprldmany_f020_160D}\\
\includegraphics[width=0.2\textwidth]{pprldmany-single-functions/pprldmany_f021_160D}&
\includegraphics[width=0.2\textwidth]{pprldmany-single-functions/pprldmany_f022_160D}&
\includegraphics[width=0.2\textwidth]{pprldmany-single-functions/pprldmany_f023_160D}&
\includegraphics[width=0.2\textwidth]{pprldmany-single-functions/pprldmany_f024_160D}
\end{tabular}
 \caption{\label{fig:ECDFsingleOne}
	\bbobecdfcaptionsinglefunctionssingledim{160}
}
\end{figure*}


%%%%%%%%%%%%%%%%%%%%%%%%%%%%%%%%%%%%%%%%%%%%%%%%%%%%%%%%%%%%%%%%%%%%%%%%%%%%%%%
%%%%%%%%%%%%%%%%%%%%%%%%%%%%%%%%%%%%%%%%%%%%%%%%%%%%%%%%%%%%%%%%%%%%%%%%%%%%%%%
 
% Table showing the expected runtime (ERT in number of function
% evaluations) for functions $f_1$--$f_{24}$ for dimension 80.

%%%%%%%%%%%%%%%%%%%%%%%%%%%%%%%%%%%%%%%%%%%%%%%%%%%%%%%%%%%%%%%%%%%%%%%%%%%%%%%
\begin{table*}\tiny
%\hfill80-D\hfill~\\[1ex]
{\normalsize \color{red}
\ifthenelse{\isundefined{\algorithmG}}{}{more than 6 algorithms: please split the tables below by hand until it fits to the page limits}
}
\mbox{\begin{minipage}[t]{0.495\textwidth}
\centering
\pptablesheader
\input{\bbobdatapath\algsfolder pptables_f001_80D} 
\input{\bbobdatapath\algsfolder pptables_f002_80D}
\input{\bbobdatapath\algsfolder pptables_f003_80D}
\input{\bbobdatapath\algsfolder pptables_f004_80D}
\input{\bbobdatapath\algsfolder pptables_f005_80D}
\input{\bbobdatapath\algsfolder pptables_f006_80D}
\input{\bbobdatapath\algsfolder pptables_f007_80D}
\input{\bbobdatapath\algsfolder pptables_f008_80D}
\input{\bbobdatapath\algsfolder pptables_f009_80D}
\input{\bbobdatapath\algsfolder pptables_f010_80D}
\input{\bbobdatapath\algsfolder pptables_f011_80D}
\input{\bbobdatapath\algsfolder pptables_f012_80D}
\end{tabularx}
\end{minipage}
\hspace{0.002\textwidth}
\begin{minipage}[t]{0.499\textwidth}\tiny
\centering
\pptablesheader
\input{\bbobdatapath\algsfolder pptables_f013_80D}
\input{\bbobdatapath\algsfolder pptables_f014_80D}
\input{\bbobdatapath\algsfolder pptables_f015_80D}
\input{\bbobdatapath\algsfolder pptables_f016_80D}
\input{\bbobdatapath\algsfolder pptables_f017_80D}
\input{\bbobdatapath\algsfolder pptables_f018_80D}
\input{\bbobdatapath\algsfolder pptables_f019_80D}
\input{\bbobdatapath\algsfolder pptables_f020_80D}
\input{\bbobdatapath\algsfolder pptables_f021_80D}
\input{\bbobdatapath\algsfolder pptables_f022_80D}
\input{\bbobdatapath\algsfolder pptables_f023_80D}
\input{\bbobdatapath\algsfolder pptables_f024_80D}
\end{tabularx}
\end{minipage}}

\caption{\label{tab:ERTs80}
\bbobpptablesmanylegend{dimension $80$} \cocoversion
}
\end{table*}

%%%%%%%%%%%%%%%%%%%%%%%%%%%%%%%%%%%%%%%%%%%%%%%%%%%%%%%%%%%%%%%%%%%%%%%%%%%%%%%
%%%%%%%%%%%%%%%%%%%%%%%%%%%%%%%%%%%%%%%%%%%%%%%%%%%%%%%%%%%%%%%%%%%%%%%%%%%%%%%
 
% Table showing the expected runtime (ERT in number of function
% evaluations) for functions $f_1$--$f_{24}$ for dimension 320.


%%%%%%%%%%%%%%%%%%%%%%%%%%%%%%%%%%%%%%%%%%%%%%%%%%%%%%%%%%%%%%%%%%%%%%%%%%%%%%%
\begin{table*}\tiny
%\hfill80-D\hfill~\\[1ex]
{\normalsize \color{red}
\ifthenelse{\isundefined{\algorithmG}}{}{more than 6 algorithms: please split the tables below by hand until it fits to the page limits}
}
\mbox{\begin{minipage}[t]{0.495\textwidth}
\centering
\pptablesheader
\input{\bbobdatapath\algsfolder pptables_f001_320D} 
\input{\bbobdatapath\algsfolder pptables_f002_320D}
\input{\bbobdatapath\algsfolder pptables_f003_320D}
\input{\bbobdatapath\algsfolder pptables_f004_320D}
\input{\bbobdatapath\algsfolder pptables_f005_320D}
\input{\bbobdatapath\algsfolder pptables_f006_320D}
\input{\bbobdatapath\algsfolder pptables_f007_320D}
\input{\bbobdatapath\algsfolder pptables_f008_320D}
\input{\bbobdatapath\algsfolder pptables_f009_320D}
\input{\bbobdatapath\algsfolder pptables_f010_320D}
\input{\bbobdatapath\algsfolder pptables_f011_320D}
\input{\bbobdatapath\algsfolder pptables_f012_320D}
\end{tabularx}
\end{minipage}
\hspace{0.002\textwidth}
\begin{minipage}[t]{0.499\textwidth}\tiny
\centering
\pptablesheader
\input{\bbobdatapath\algsfolder pptables_f013_320D}
\input{\bbobdatapath\algsfolder pptables_f014_320D}
\input{\bbobdatapath\algsfolder pptables_f015_320D}
\input{\bbobdatapath\algsfolder pptables_f016_320D}
\input{\bbobdatapath\algsfolder pptables_f017_320D}
\input{\bbobdatapath\algsfolder pptables_f018_320D}
\input{\bbobdatapath\algsfolder pptables_f019_320D}
\input{\bbobdatapath\algsfolder pptables_f020_320D}
\input{\bbobdatapath\algsfolder pptables_f021_320D}
\input{\bbobdatapath\algsfolder pptables_f022_320D}
\input{\bbobdatapath\algsfolder pptables_f023_320D}
\input{\bbobdatapath\algsfolder pptables_f024_320D}
\end{tabularx}
\end{minipage}}

\caption{\label{tab:ERTs320}
\bbobpptablesmanylegend{dimension $320$} \cocoversion
}
\end{table*}


}{} % end of all that comes for 2 or 3+ algorithms



\ifthenelse{\numofalgs = 2}{


%%%%%%%%%%%%%%%%%%%%%%%%%%%%%%%%%%%%%%%%%%%%%%%%%%%%%%%%%%%%%%%%%%%%%%%%%%%%%%%
%%%%%%%%%%%%%%%%%%%%%%%%%%%%%%%%%%%%%%%%%%%%%%%%%%%%%%%%%%%%%%%%%%%%%%%%%%%%%%%
 
% Scatter plots per function.

%%%%%%%%%%%%%%%%%%%%%%%%%%%%%%%%%%%%%%%%%%%%%%%%%%%%%%%%%%%%%%%%%%%%%%%%%%%%%%%

%%%%%%%%%%%%%%%%%%%%%%%%%%%%%%%%%%%%%%%%%%%%%%%%%%%%%%%%%%%%%%%%%%%%%%%%%%%%%%%
%%%%%%%%%%%%%%%%%%%%%%%%%%%%%%%%%%%%%%%%%%%%%%%%%%%%%%%%%%%%%%%%%%%%%%%%%%%%%%%
%%%%%%%%%%%%%%%%%%%%%%%%%%%%%%%%%%%%%%%%%%%%%%%%%%%%%%%%%%%%%%%%%%%%%%%%%%%%%%%

\begin{figure*}
\begin{tabular}{*{4}{@{}c@{}}}
    \includegraphics[height=0.2\textwidth]{ppscatter_f001}&
    \includegraphics[height=0.2\textwidth]{ppscatter_f002}&
    \includegraphics[height=0.2\textwidth]{ppscatter_f003}&
    \includegraphics[height=0.2\textwidth]{ppscatter_f004}\\[-0.6em]
    \includegraphics[height=0.2\textwidth]{ppscatter_f005}&
    \includegraphics[height=0.2\textwidth]{ppscatter_f006}&
    \includegraphics[height=0.2\textwidth]{ppscatter_f007}&
    \includegraphics[height=0.2\textwidth]{ppscatter_f008}\\[-0.6em]
    \includegraphics[height=0.2\textwidth]{ppscatter_f009}&
    \includegraphics[height=0.2\textwidth]{ppscatter_f010}&
    \includegraphics[height=0.2\textwidth]{ppscatter_f011}&
    \includegraphics[height=0.2\textwidth]{ppscatter_f012}\\[-0.6em]
    \includegraphics[height=0.2\textwidth]{ppscatter_f013}&
    \includegraphics[height=0.2\textwidth]{ppscatter_f014}&
    \includegraphics[height=0.2\textwidth]{ppscatter_f015}&
    \includegraphics[height=0.2\textwidth]{ppscatter_f016}\\[-0.6em]
    \includegraphics[height=0.2\textwidth]{ppscatter_f017}&
    \includegraphics[height=0.2\textwidth]{ppscatter_f018}&
    \includegraphics[height=0.2\textwidth]{ppscatter_f019}&
    \includegraphics[height=0.2\textwidth]{ppscatter_f020}\\[-0.6em]
    \includegraphics[height=0.2\textwidth]{ppscatter_f021}&
    \includegraphics[height=0.2\textwidth]{ppscatter_f022}&
    \includegraphics[height=0.2\textwidth]{ppscatter_f023}&
    \includegraphics[height=0.2\textwidth]{ppscatter_f024}
\end{tabular}
\caption{\label{fig:scatterplots}
\bbobppscatterlegend{$f_1$--$f_{24}$}
}
\end{figure*}



} % end of 2 algorithms template




%%%%%%%%%%%%%%%%%%%%%%%%%%%%%%%%%%%%%%%%%%%%%%%%%%%%%%%%%%%%%%%%%%%%%%%%%%%%%%%
%%%%%%%%%%%%%%%%%%%%%%%%%%%%%%%%%%%%%%%%%%%%%%%%%%%%%%%%%%%%%%%%%%%%%%%%%%%%%%%

\bibliographystyle{ACM-Reference-Format}
\bibliography{bbob}  % bbob.bib is the name of the Bibliography in this case

%%%%%%%%%%%%%%%%%%%%%%%%%%%%%%%%%%%%%%%%%%%%%%%%%%%%%%%%%%%%%%%%%%%%%%%%%%%%%%%%%%%%%%%%%%%
\end{document}
